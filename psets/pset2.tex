\documentclass[10pt,oneside]{article}
\usepackage{amsmath} 
\usepackage{graphicx} 
\usepackage{subcaption} 
\usepackage{amsfonts}
\usepackage{amssymb} 

\newcommand{\tr}{\operatorname{trace}}
\newcommand{\vecm}{\operatorname{vec}}

\newcommand{\dotstar}{\operatorname{.*}}

\usepackage{polyglossia}




\usepackage[
  backend=biber
]{biblatex}
\addbibresource{biblio.bib}


\usepackage[
  letterpaper,
  left=1cm,
  right=1cm,
  top=1.5cm,
  bottom=1.5cm
]{geometry}


\usepackage[
  final,
  unicode,
  colorlinks=true,
  citecolor=blue,
  linkcolor=blue,
  plainpages=false,
  urlcolor=blue,
  pdfpagelabels=true,
  pdfsubject={Cálculo},
  pdfauthor={José Doroteo Arango Arámbula},
  pdftitle={Tarea 1},
  pdfkeywords={UNAM, FES Acatlán, 2021-I}
]{hyperref}

\usepackage{booktabs}


\usepackage{algpseudocode}
\usepackage{algorithm} 
\floatname{algorithm}{Algoritmo}

\usepackage{enumitem}

\usepackage{lastpage}
\usepackage{fancyhdr}
\fancyhf{}
\pagestyle{fancy}
\fancyhf{}
\fancyhead[L]{MIT IAP January 2024} 
\fancyhead[C]{Matrix Calculus} 
\fancyhead[R]{Profs. Edelman and Johnson} 
\fancyfoot[R]{}
% https://tex.stackexchange.com/questions/227/how-can-i-add-page-of-on-my-document
\fancyfoot[C]{\thepage\ of \pageref*{LastPage}}
%\fancyfoot[C]{ of }
\fancyfoot[L]{} 
\renewcommand{\headrulewidth}{2pt} 
\renewcommand{\footrulewidth}{2pt}

\usepackage[newfloat=true]{minted} 

\author{} 
\title{Homework 2}

\date{\today}


\DeclareCaptionFormat{mitedFormat}{%
    \textbf{#1#2}#3}
\DeclareCaptionStyle{minetdStyle}{skip=0cm,width=.85\textwidth,justification=centering,
  font=footnotesize,singlelinecheck=off,format=mitedFormat,labelsep=space}
\newenvironment{mintedCode}{\captionsetup{type=listing,style=minetdStyle}}{}

\usepackage{dirtytalk} 

\SetupFloatingEnvironment{listing}{}

\usepackage[parfill]{parskip}

\usepackage{csquotes} 

\begin{document}
\maketitle
\thispagestyle{fancy} 

{\bf Please submit your HW on Canvas; include a PDF printout of any code and results, clearly labeled, e.g. from a Jupyter notebook.  For coding problems, we recommend using Julia, but you can use other languages if you wish. It is due Friday February~2nd by 11:59pm EST.  }

\subsection*{Problem 1}

Continue reading the draft course notes (linked from \url{https://github.com/mitmath/matrixcalc/}).   Find another place that you found confusing, in a different chapter from your answer in homework~1, and write a paragraph explaining the source of your confusion and (ideally) suggesting a possible improvement.

(Any other corrections/comments are welcome, too.)

\subsection*{Problem 2}

The \href{https://github.com/mitmath/matrixcalc/blob/a2c2d3c5f9269f501fa03cd14e3fbe8d219df211/notes/Finite\%20difference\%20checks.ipynb}{course notebook on finite differences}
includes, without derivation, a mysterious four-line Julia function
called \texttt{stencil} that can compute finite-difference rules for
an arbitrary number of points. In particular, if you want to compute
the $m$-th derivative of a smooth (analytic) scalar function $f(x)$
at $x_{0}$, it returns the weights $w_{k}$ of an $n$-point ($n>m$)
finite-difference rule from evaluating $f$ at points $x_{k}$ for
$k=1\ldots n$:
\[
f^{(m)}(x_{0})\approx\sum_{k=1}^{n}w_{k}f(x_{k})
\]
by solving the system of equations $Aw=e_{m+1}$, where $e_{j}\in\mathbb{R}^{n}$
is the Cartesian unit vector in the $j$-th direction and $A$ is
an $n\times n$ matrix with entries $A_{ij}=\frac{(x_{j}-x_{0})^{i-1}}{(i-1)!}$
. Here, you will analyze and derive this technique.
\begin{enumerate}
\item Let $x_{0}=0$. According to the notes, you can then compute $f^{(m)}(y)\approx\frac{1}{h^{m}}\sum_{k=1}^{n}w_{k}f(y+hx_{k})$
for an arbitrary point $y$ and an arbitrary step-size scaling factor
$h$ (which can be made smaller and smaller to reduce truncation errors).
Derive this formula (via the chain rule).
\item Evaluate the \texttt{stencil} function (or its equivalent in another
language if you want to re-implement it) for $x_{0}=0$ and $x=[0,1]$
with $m=1$. Check that the resulting $w$ corresponds to the familiar
forward-difference approximation from class. (You could alternatively
solve $Aw=e_{m+1}$ analytically here, since it is $2\times2$.) In
Julia, you can pass \texttt{0//1} for $x_{0}$ and it will return
exact rational weights.
\item Now evaluate it for $x_{0}=0$ (\texttt{0//1} in Julia for exact results)
and $x=[0,1,2,3]$ with $m=1$, i.e. using $n=4$ equally spaced points
$\ge x_{0}$. Use the resulting weights, in the formula scaled by
$h$ as above, to approximate the derivative $f'(1)$ for $f(x)=\sin(x)$,
and plot the relative error (compared to the exact derivative) as
a function of $h$ on a log–log scale, similar to the course notebook.
What power law in $h$ does the truncation error (approximately) seem
to follow? That is, what is the “order of accuracy”?
\item Derive the stencil equation $Aw=e_{m+1}$ above: write out the first
$n-1$ terms of the Taylor series for $f(x_{0}+\delta x)$, and try
to find a linear combination of this series evaluated at $\delta x=x_{k}-x_{0}$
for $k=1\ldots n$ in such a way that you obtain $f^{(m)}(x_{0})$.
\end{enumerate}

\subsection*{Problem 3}

Consider the following system $g(x,p)=0$ of two nonlinear equations
in two variables $x\in\mathbb{R}^{2}$, parameterized by three parameters
$p\in\mathbb{R}^{3}$:
\[
g(x,p)=\left(\begin{array}{c}
p_{1}x_{1}^{2}-x_{2}\\
x_{1}x_{2}-p_{2}x_{2}+p_{3}
\end{array}\right)=\left(\begin{array}{c}
0\\
0
\end{array}\right).
\]
For $p=[1,2,1]$ this has an exact solution $x=[1,1]$.
\begin{enumerate}
\item What are the Jacobian matrices $\frac{\partial g}{\partial x}$ and
$\frac{\partial g}{\partial p}$? (That is, as defined in class, the
linear operators such that $dg=\frac{\partial g}{\partial x}dx+\frac{\partial g}{\partial p}dp$
for any change $dx$ and $dp$, to first order.)
\item In Julia (or Python etc.), implement Newton's method to solve $g(x,p)=0$
from a given starting guess~$x$. Using $p=[1,2,1]$, start your
Newton iteration at $x=[1.2,1.3]$ and show that it converges rapidly
to $x=[1,1]$ (it should converge to machine precision in $<10$ steps).
\item Now, consider the function $f(p)=\Vert x(p)\Vert$, where the ``implicit
function''\footnote{This $x(p)$ can be defined uniquely in some neighborhood of a root
like the one above, thanks to the implicit-function theorem.} $x(p)$ is a solution of $g(x,p)=0$. Given a solution $x(p)$ for
some $p$, explain how to compute $\nabla f$ (see the adjoint-method
notes from lecture~5). Implement this algorithm in Julia (etc.),
and validate it against a finite-difference approximation for $p=[1,2,1]$,
$x(p)=[1,1]$, and a random small $\delta p$ (solving for $x(p+\delta p)$
by Newton's method starting from $x(p)$).
\end{enumerate}

\subsection*{Problem 4}
\begin{enumerate}
\item Suppose that $f(A)$ is a function that maps (real) $m\times m$ matrices
to $m\times m$ matrices, and its derivative is the linear operator
$f'(A)[dA]$. For the Frobenius inner product $\langle X,Y\rangle=\tr(X^{T}Y)$,
it turns out that we typically have 
\[
\left\langle X,f'(A)[Y]\right\rangle =\left\langle f'(A^{T})[X],Y\right\rangle ,
\]
which conceptually corresponds to “transposing” the linear operator
$f'(A)^{T}=f'(A^{T})$. Your job is to show this.
\begin{enumerate}
\item Show this for $f(A)=A^{n}$ for any $n\ge0$. (Hint: From the product
rule, it is easy to see that $f'(A)[dA]=\sum_{k=0}^{n-1}A^{k}dA\,A^{n-1-k}$;
we've already seen this explicitly for several $n$. Combine this
with the cyclic rule for the trace.)\\
\\
It immediately follows that this identity also works for any $f(A)$
described by a Taylor series in~$A$ (any ``analytic'' $f$), such
as $e^{A}$.
\item Show this for $f(A)=A^{-1}$.\\
\\
(You can then compose the above cases to show that it works for any
$f(A)=p(A)q(A)^{-1}$ for any polynomials~$p$ and~$q$, i.e.~for
any rational function of $A$. You need not do this, however.)
\end{enumerate}
\item Consider the function $f(A)=\det(A + \exp(A))$.
\begin{enumerate}
\item Write $f'(A)[dA]$ in terms of $\exp'(A)[dA]$ and other standard matrix operations. (You learned how to
compute $\exp'$ in pset~1.)
\item Using the identity from the previous part, write $\nabla f$ in a
way that can be evaluated efficiently (“reverse mode”) using
only one or two evaluations of $\exp$ (and/or $\exp'$) and $\det$,
independent of the size of $A$.
\end{enumerate}
\item Check your answer from the previous part in Julia (or Python etc.):
choose a random $5\times5$ \texttt{A=randn(5,5)} and a random small
\texttt{dA=randn(5,5){*}1e-8}, compute $df=f(A+dA)-f(A)$ and $\nabla f$
(at $A$), and verify that $df\approx\langle\nabla f,dA\rangle$.
Compute $\exp'(A)[dA]$ using the same technique as in pset~1.
\end{enumerate}
\end{document}
